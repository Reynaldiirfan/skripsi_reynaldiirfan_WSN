%versi 3 (18-12-2016)
\chapter{Kode Program Aplikasi Pengguna (Website)}
\label{lamp:C}

%terdapat 2 cara untuk memasukkan kode program
% 1. menggunakan perintah \lstinputlisting (kode program ditempatkan di folder yang sama dengan file ini)
% 2. menggunakan environment lstlisting (kode program dituliskan di dalam file ini)
% Perhatikan contoh yang diberikan!!
%
% untuk keduanya, ada parameter yang harus diisi:
% - language: bahasa dari kode program (pilihan: Java, C, C++, PHP, Matlab, C#, HTML, R, Python, SQL, dll)
% - caption: nama file dari kode program yang akan ditampilkan di dokumen akhir
%
% Perhatian: Abaikan warning tentang textasteriskcentered!!
%

\begin{lstlisting}[language=PHP, caption=skripsi\_header.blade.php]
<!DOCTYPE html>
<html lang="en">
<head>
    <meta charset="UTF-8">
    <meta name="viewport" content="width=device-width, initial-scale=1.0, shrink-to-fit=no">

    <link rel="stylesheet" href="https://maxcdn.bootstrapcdn.com/bootstrap/4.0.0/css/bootstrap.min.css" integrity="sha384-Gn5384xqQ1aoWXA+058RXPxPg6fy4IWvTNh0E263XmFcJlSAwiGgFAW/dAiS6JXm" crossorigin="anonymous">

    <script src="https://code.jquery.com/jquery-3.2.1.slim.min.js" integrity="sha384-KJ3o2DKtIkvYIK3UENzmM7KCkRr/rE9/Qpg6aAZGJwFDMVNA/GpGFF93hXpG5KkN" crossorigin="anonymous"></script>
    <script src="https://cdnjs.cloudflare.com/ajax/libs/popper.js/1.12.9/umd/popper.min.js" integrity="sha384-ApNbgh9B+Y1QKtv3Rn7W3mgPxhU9K/ScQsAP7hUibX39j7fakFPskvXusvfa0b4Q" crossorigin="anonymous"></script>
    <script src="https://maxcdn.bootstrapcdn.com/bootstrap/4.0.0/js/bootstrap.min.js" integrity="sha384-JZR6Spejh4U02d8jOt6vLEHfe/JQGiRRSQQxSfFWpi1MquVdAyjUar5+76PVCmYl" crossorigin="anonymous"></script>
    
    <title>Home</title>
    <link href="{{ asset('css/skripsi_mainstyle.css') }}" rel="stylesheet" type="text/css" >
</head>
<style>
</style>

<body>
    
    <ul style="margin-top: 4px;">
        <li><img src="assets/logo2.png" style="height: 38px;margin-top: 8px;;"></li>
        <li><a href="/">Halaman Utama</a></li>
        <li><a href="/check-status">Check Status</a></li>
        <li><a href="/sensing">Sensing</a></li>


        <li style="float: right;"><a href="/cara-pakai">Cara Pakai</a></li>
        <li style="float: right;"><a href="/print-sensing">Print Sensing</a></li>
        
    </ul>
    
</body>
</html>
\end{lstlisting}

\begin{lstlisting}[language=PHP, caption=skripsi\_home.blade.php]
<!DOCTYPE html>
<html lang="en">
<head>
    <meta charset="UTF-8">
    <meta name="viewport" content="width=device-width, initial-scale=1.0, shrink-to-fit=no">
    <link rel="icon" href="{{ asset('assets/unpar.png') }}">
    <link rel="stylesheet" href="https://maxcdn.bootstrapcdn.com/bootstrap/4.0.0/css/bootstrap.min.css" integrity="sha384-Gn5384xqQ1aoWXA+058RXPxPg6fy4IWvTNh0E263XmFcJlSAwiGgFAW/dAiS6JXm" crossorigin="anonymous">

    <script src="https://code.jquery.com/jquery-3.2.1.slim.min.js" integrity="sha384-KJ3o2DKtIkvYIK3UENzmM7KCkRr/rE9/Qpg6aAZGJwFDMVNA/GpGFF93hXpG5KkN" crossorigin="anonymous"></script>
    <script src="https://cdnjs.cloudflare.com/ajax/libs/popper.js/1.12.9/umd/popper.min.js" integrity="sha384-ApNbgh9B+Y1QKtv3Rn7W3mgPxhU9K/ScQsAP7hUibX39j7fakFPskvXusvfa0b4Q" crossorigin="anonymous"></script>
    <script src="https://maxcdn.bootstrapcdn.com/bootstrap/4.0.0/js/bootstrap.min.js" integrity="sha384-JZR6Spejh4U02d8jOt6vLEHfe/JQGiRRSQQxSfFWpi1MquVdAyjUar5+76PVCmYl" crossorigin="anonymous"></script>
    
    <title>Halaman Utama | Skripsi</title>
    <link href="{{ asset('css/skripsi_mainstyle.css') }}" rel="stylesheet" type="text/css" >
</head>
<style>
    body{
        background-image: url("assets/home_bg.jpg");
        background-size: cover;
        padding-left: 99px;
        padding-right: 108px;
    }
</style>
<body>
    @include('skripsi_header')

    <h1>Aplikasi Pemantauan<br> 
        Kualitas Tanah Sawah<br>
        Berbasis WSN
    </h1>

    <hr>

    <h2>
        Reynaldi Irfan A <br>
        2016730045
    </h2>

    <button class="button button4" onclick="location.href='{{ url('sensing') }}'">Let's Get Started</button>
</body>
</html>
\end{lstlisting}

\begin{lstlisting}[language=PHP, caption=skripsi\_checkstatus.blade.php]
<!DOCTYPE html>
<html lang="en">
<head>
    <meta charset="UTF-8">
    <meta name="viewport" content="width=device-width, initial-scale=1.0, shrink-to-fit=no">
    <link rel="icon" href="{{ asset('assets/unpar.png') }}">
    <link rel="stylesheet" href="https://maxcdn.bootstrapcdn.com/bootstrap/4.0.0/css/bootstrap.min.css" integrity="sha384-Gn5384xqQ1aoWXA+058RXPxPg6fy4IWvTNh0E263XmFcJlSAwiGgFAW/dAiS6JXm" crossorigin="anonymous">

    <script src="https://code.jquery.com/jquery-3.2.1.slim.min.js" integrity="sha384-KJ3o2DKtIkvYIK3UENzmM7KCkRr/rE9/Qpg6aAZGJwFDMVNA/GpGFF93hXpG5KkN" crossorigin="anonymous"></script>
    <script src="https://cdnjs.cloudflare.com/ajax/libs/popper.js/1.12.9/umd/popper.min.js" integrity="sha384-ApNbgh9B+Y1QKtv3Rn7W3mgPxhU9K/ScQsAP7hUibX39j7fakFPskvXusvfa0b4Q" crossorigin="anonymous"></script>
    <script src="https://maxcdn.bootstrapcdn.com/bootstrap/4.0.0/js/bootstrap.min.js" integrity="sha384-JZR6Spejh4U02d8jOt6vLEHfe/JQGiRRSQQxSfFWpi1MquVdAyjUar5+76PVCmYl" crossorigin="anonymous"></script>

    <title>Check Status | Skripsi</title>
    <link href="{{ asset('css/skripsi_mainstyle.css') }}" rel="stylesheet" type="text/css" >
</head>
<style>
    body{
        background-image: url("assets/sensing_bg.jpg") ;
        background-size: 100%;
        padding-left: 99px;
        padding-right: 108px;
        background-repeat: no-repeat;
    }
</style>
<body>
    @include('skripsi_header')
    <br>
    <table style="margin-top: 15px;float: left; width:45%;" id="table1">
        @foreach ($check1 as $check1)
        <tr style="background-color: rgba(0, 0, 0, 0.5)">
          <th>Variable</th>
          <th>Value</th> 
        </tr>
        <tr>
            <td>Nama Node</td>
            <td>{{ $check1->nama_node}}</td>
          </tr>
        <tr>
          <td>Status Node</td>
          <td id="n1_aktif">{{ $check1->status_node}}</td>
        </tr>
	<tr>
          <td>Status Sensing</td>
          <td id="n1_sensing">{{ $check1->status_sensing}}</td>
        </tr>
        @endforeach
    </table>

    <table style="margin-top: 15px;float: left; width:45%;" id="table2">
        @foreach ($check2 as $check2)
        <tr style="background-color: rgba(0, 0, 0, 0.5)">
          <th>Variable</th>
          <th>Value</th> 
        </tr>
        <tr>
            <td>Nama Node</td>
            <td>{{ $check2->nama_node}}</td>
          </tr>
        <tr>
          <td>Status Node</td>
          <td id="n2_aktif">{{ $check2->status_node}}</td>
        </tr>
	<tr>
          <td>Status Sensing</td>
          <td id="n2_sensing">{{ $check2->status_sensing}}</td>
        </tr>
        @endforeach
    </table>

    <br>

    <table style="margin-top: 24px;float: left; width:45%;" id="table3">
        @foreach ($check3 as $check3)
        <tr style="background-color: rgba(0, 0, 0, 0.5)">
          <th>Variable</th>
          <th>Value</th> 
        </tr>
        <tr>
            <td>Nama Node</td>
            <td>{{ $check3->nama_node}}</td>
          </tr>
        <tr>
          <td>Status Node</td>
          <td id="n3_aktif">{{ $check3->status_node}}</td>
        </tr>
	<tr>
          <td>Status Sensing</td>
          <td id="n3_sensing">{{ $check3->status_sensing}}</td>
        </tr>
        @endforeach
    </table>

    <table style="margin-top: 24px;float: left; width:45%;" id="table4">
        @foreach ($check4 as $check4)
        <tr style="background-color: rgba(0, 0, 0, 0.5)">
          <th>Variable</th>
          <th>Value</th> 
        </tr>
        <tr>
            <td>Nama Node</td>
            <td>{{ $check4->nama_node}}</td>
          </tr>
        <tr>
          <td>Status Node</td>
          <td id="n4_aktif">{{ $check4->status_node}}</td>
        </tr>
	<tr>
          <td>Status Sensing</td>
          <td id="n4_sensing">{{ $check4->status_sensing}}</td>
        </tr>
        @endforeach
    </table>

    <br>

    <table style="margin-top: 24px;float: left;margin-left:26%; " id="table5">
        @foreach ($check5 as $check5)
        <tr style="background-color: rgba(0, 0, 0, 0.5)">
          <th>Variable</th>
          <th>Value</th> 
        </tr>
        <tr>
            <td>Nama Node</td>
            <td>{{ $check5->nama_node}}</td>
          </tr>
        <tr>
          <td>Status Node</td>
          <td id="n5_aktif">{{ $check5->status_node}}</td>
        </tr>
        @endforeach
    </table>
    
    <div style="margin-left:15px;margin-top: 418px;">
        <img src="assets/back.png"><a href="/" style="color: white;">Kembali</a>
    </div>
    
    <form method="post" action="/check-status/update/">
        {{ csrf_field() }}
        {{ method_field('PUT') }}
        <button class="button button4" style="visibility:hidden;" onclick="changeButtonText()" value="online" id="sensing_button" style="margin-top: 140px;margin-right: 15px;">Update</button>
    </form>
    
</body>

<script>
      var buttonSensing = document.getElementById('sensing_button').value;
  var statusSensing = "true";
  
    
    var node1Stat = document.getElementById('n1_aktif').textContent;
    var node1Sens = document.getElementById('n1_sensing').textContent;
    
    var node2Stat = document.getElementById('n2_aktif').textContent;
    var node2Sens = document.getElementById('n2_sensing').textContent;

    var node3Stat = document.getElementById('n3_aktif').textContent;
    var node3Sens = document.getElementById('n3_sensing').textContent;

    var node4Stat = document.getElementById('n4_aktif').textContent;
    var node4Sens = document.getElementById('n4_sensing').textContent;

    var node5Stat = document.getElementById('n5_aktif').textContent;

    //CHECK STATUS NODE
    if(node1Stat!='Online'){
      document.getElementById('n1_aktif').style.color="yellow";
      document.getElementById('n1_sensing').style.color="yellow";
    }
    if(node2Stat!='Online'){
      document.getElementById('n2_aktif').style.color="yellow";
      document.getElementById('n2_sensing').style.color="yellow";
    }
    if(node3Stat!='Online'){
      document.getElementById('n3_aktif').style.color="yellow";
      document.getElementById('n3_sensing').style.color="yellow";
    }
    if(node4Stat!='Online'){
      document.getElementById('n4_aktif').style.color="yellow";
      document.getElementById('n4_sensing').style.color="yellow";
    }
    if(node5Stat!='Online'){
      document.getElementById('n5_aktif').style.color="yellow";
    }

    //CHECK STATUS SENSING
    if(node1Sens!='Sensing'){
      document.getElementById('n1_sensing').style.color="yellow";
    }
    if(node2Sens!='Sensing'){
      document.getElementById('n2_sensing').style.color="yellow";
    }
    if(node3Sens!='Sensing'){
      document.getElementById('n3_sensing').style.color="yellow";
    }
    if(node4Sens!='Sensing'){
      document.getElementById('n4_sensing').style.color="yellow";
    }
    
    
    //TIMER UPDATE 30 sec STATUS NODE
    if(node1Stat=='Online'){
        setTimeout(setStatusN1, 30000);
    }
    if(node2Stat=='Online'){
        setTimeout(setStatusN2, 30000);
    }
    if(node3Stat=='Online'){
        setTimeout(setStatusN3, 30000);
    }
    if(node4Stat=='Online'){
        setTimeout(setStatusN4, 30000);
    }
        
    
    //GANTI STATUS    
    function setStatusN1() {
        node1Stat='Offline'
        node1Sens!='Not Sensing'
    }
    function setStatusN2() {
        node2Stat='Offline'
        node1Sens!='Not Sensing'
    }
    function setStatusN3() {
        node3Stat='Offline'
        node1Sens!='Not Sensing'
    }
    function setStatusN4() {
        node4Stat='Offline'
        node1Sens!='Not Sensing'
    }
    
</script>
\end{lstlisting}


\begin{lstlisting}[language=PHP, caption=skripsi\_sensing.blade.php]

<!DOCTYPE html>
<html lang="en">
<head>
    <meta charset="UTF-8">
    <meta name="viewport" content="width=device-width, initial-scale=1.0, shrink-to-fit=no">
    <link rel="icon" href="{{ asset('assets/unpar.png') }}">
    <link rel="stylesheet" href="https://maxcdn.bootstrapcdn.com/bootstrap/4.0.0/css/bootstrap.min.css" integrity="sha384-Gn5384xqQ1aoWXA+058RXPxPg6fy4IWvTNh0E263XmFcJlSAwiGgFAW/dAiS6JXm" crossorigin="anonymous">

    <script src="https://code.jquery.com/jquery-3.2.1.slim.min.js" integrity="sha384-KJ3o2DKtIkvYIK3UENzmM7KCkRr/rE9/Qpg6aAZGJwFDMVNA/GpGFF93hXpG5KkN" crossorigin="anonymous"></script>
    <script src="https://cdnjs.cloudflare.com/ajax/libs/popper.js/1.12.9/umd/popper.min.js" integrity="sha384-ApNbgh9B+Y1QKtv3Rn7W3mgPxhU9K/ScQsAP7hUibX39j7fakFPskvXusvfa0b4Q" crossorigin="anonymous"></script>
    <script src="https://maxcdn.bootstrapcdn.com/bootstrap/4.0.0/js/bootstrap.min.js" integrity="sha384-JZR6Spejh4U02d8jOt6vLEHfe/JQGiRRSQQxSfFWpi1MquVdAyjUar5+76PVCmYl" crossorigin="anonymous"></script>

    <title>Sensing | Skripsi</title>
    <link href="{{ asset('css/skripsi_mainstyle.css') }}" rel="stylesheet" type="text/css" >
</head>
{{-- <script src="https://js.pusher.com/6.0/pusher.min.js"></script>
<script>
    // Enable pusher logging - don't include this in production
    Pusher.logToConsole = true;

    var pusher = new Pusher('c559846b0d74d3af1611', {
      cluster: 'ap1'
    });

    var channel = pusher.subscribe('my-channel');
    channel.bind('my-event', function(data) {
      alert(JSON.stringify(data));
    });
  </script> TADINYA UNTUK WEBSOCKET--}}  
<style>
    body{
        background-image: url("assets/sensing_bg.jpg") ;
        background-size: cover;
        padding-left: 99px;
        padding-right: 108px;
    }
</style>
<body>
  @include('skripsi_header')

    <table style="margin-top: 15px;float: left; width:45%;" id="table1">
      @foreach ($nodes1 as $node1)
      <tr style="background-color: rgba(0, 0, 0, 0.5)">
        <th>Variable</th>
        <th>Value</th> 
      </tr>
      <tr>
          <td>Nama Node</td>
          <td id="namanode1">{{ $node1->nama_node }}</td>
        </tr>
      <tr>
        <td>Keasaman (pH)</td>
        <td id="node1_ph">{{ $node1->ph_tanah }}</td>
      </tr>
      <tr>
        <td>Kelembaban Tanah</td>
        <td id="node1_lembab">{{ $node1->kelembaban_tanah }} %</td>
      </tr>
      <tr>
        <td>Suhu Tanah</td>
        <td id="node1_suhuTh">{{ $node1->suhu_tanah }} <span>&#8451;</span></td>
      </tr>
      <tr>
        <td>Suhu Udara</td>
        <td id="node1_suhu">{{ $node1->suhu_udara }} <span>&#8451;</span></td>
      </tr>
      <tr>
        <td>Waktu Sensing</td>
        <td id="node1_waktu">{{ $node1->waktu_sensing }}</td>
      </tr>
      @endforeach
    </table>

    <table style="margin-top: 15px;float: left; width:45%;" id="table2">
      @foreach ($nodes2 as $node2)
      <tr style="background-color: rgba(0, 0, 0, 0.5)">
        <th>Variable</th>
        <th>Value</th> 
      </tr>
      <tr>
          <td>Nama Node</td>
          <td>{{ $node2->nama_node }}</td>
        </tr>
      <tr>
        <td>Keasaman (pH)</td>
        <td id="node2_ph">{{ $node2->ph_tanah }}</td>
      </tr>
      <tr>
        <td>Kelembaban Tanah</td>
        <td id="node2_lembab">{{ $node2->kelembaban_tanah }} %</td>
      </tr>
      <tr>
        <td>Suhu Tanah</td>
        <td id="node2_suhuTh">{{ $node2->suhu_tanah }} <span>&#8451;</span></td>
      </tr>
      <tr>
        <td>Suhu Udara</td>
        <td id="node2_suhu">{{ $node2->suhu_udara }} <span>&#8451;</span></td>
      </tr>
      <tr>
        <td>Waktu Sensing</td>
        <td id="node2_waktu">{{ $node2->waktu_sensing }} </td>
      </tr>
      @endforeach
    </table>

    <br>

    <table style="margin-top: 16px;float: left; width:45%;" id="table3">
      @foreach ($nodes3 as $node3)
        <tr style="background-color: rgba(0, 0, 0, 0.5)">
          <th>Variable</th>
          <th>Value</th> 
        </tr>
        <tr>
            <td>Nama Node</td>
            <td>{{ $node3->nama_node }}</td>
          </tr>
        <tr>
          <td>Keasaman (pH)</td>
          <td id="node3_ph">{{ $node3->ph_tanah }}</td>
        </tr>
        <tr>
          <td>Kelembaban Tanah</td>
          <td id="node3_lembab">{{ $node3->kelembaban_tanah }} %</td>
        </tr>
        <tr>
          <td>Suhu Tanah</td>
          <td id="node3_suhuTh">{{ $node3->suhu_tanah }} <span>&#8451;</span></td>
        </tr>
        <tr>
          <td>Suhu Udara</td>
          <td id="node3_suhu">{{ $node3->suhu_udara }} <span>&#8451;</span></td>
        </tr>
        <tr>
          <td>Waktu Sensing</td>
          <td id="node3_waktu">{{ $node3->waktu_sensing }} </td>
        </tr>
        @endforeach
    </table>

    <table style="margin-top: 20px;float: left; width:45%;" id="table4" >
      @foreach ($nodes4 as $node4)
      <tr style="background-color: rgba(0, 0, 0, 0.5)">
        <th>Variable</th>
        <th>Value</th> 
      </tr>
      <tr>
          <td>Nama Node</td>
          <td>{{ $node4->nama_node }}</td>
        </tr>
      <tr>
        <td>Keasaman (pH)</td>
        <td id="node4_ph">{{ $node4->ph_tanah }}</td>
      </tr>
      <tr>
        <td>Kelembaban Tanah</td>
        <td id="node4_lembab">{{ $node4->kelembaban_tanah }} %</td>
      </tr>
      <tr>
        <td>Suhu Tanah</td>
        <td id="node4_suhuTh">{{ $node4->suhu_tanah }} <span>&#8451;</span></td>
      </tr>
      <tr>
        <td>Suhu Udara</td>
        <td id="node4_suhu">{{ $node4->suhu_udara }} <span>&#8451;</span></td>
      </tr>
      <tr>
        <td>Waktu Sensing</td>
        <td id="node4_waktu">{{ $node4->waktu_sensing }} </td>
      </tr>
      @endforeach
    </table>

    <button style="visibility:hidden;" class="button button4" onclick="changeButtonText()" value="online" id="sensing_button" style="margin-top: 140px;margin-right: 15px;">Pause Alert</button>
    
    <!-- <div style="margin-left:15px;margin-top: 468px;">
      <img src="assets/back.png"><a href="/" style="color: white;">Kembali</a>
    </div> -->
    
    <script>
 

  var buttonSensing = document.getElementById('sensing_button').value;
  var statusSensing = "true";
  

  function changeButtonText(){
    if(buttonSensing=="offline"){ 
      document.getElementById('sensing_button').innerHTML = 'Stop Sensing';
      buttonSensing = "online";
      alert("Sensing akan dimulai !");
      statusSensing = "true";
      location.reload(),6000;
      return false;
      
      
    }
    else{ 
      document.getElementById('sensing_button').innerHTML = 'Mulai Sensing';
      alert("Sensing dihentikan! Klik OK untuk melanjutkan sensing");
      
      
      var table1 = document.getElementById('table1');
      var table2 = document.getElementById('table2');
      var table3 = document.getElementById('table3');
      var table = document.getElementById('table4');
      var lengthTable = table.rows.length;

      for(i=1;i<lengthTable;i++){
        var cells = table.rows.item(i).cells;
        var cells1 = table1.rows.item(i).cells;
        var cells2 = table2.rows.item(i).cells;
        var cells3 = table3.rows.item(i).cells;

        //gets amount of cells of current row
        var cellLength =cells.length;

        //loops through each cell in current row
        for(var j = 1; j < cellLength; j++){
          /* get your cell info here */
          cells.item(j).innerHTML= '-';
          cells1.item(j).innerHTML= '-';
          cells2.item(j).innerHTML= '-';
          cells3.item(j).innerHTML= '-';
        }
      }
      buttonSensing = "offline";
      statusSensing = "false";
    }
  }
  
    if(statusSensing=="true"){
        setTimeout(function() {
        location.reload();
      }, 5000);
    }

    //CHECK KLASIFIKASI PH TANAH
    var nPh1 = parseFloat(document.getElementById('node1_ph').textContent);
    var nPh2 = parseFloat(document.getElementById('node2_ph').textContent);
    var nPh3 = parseFloat(document.getElementById('node3_ph').textContent);
    //var nPh4 = parseFloat(document.getElementById('node4_ph').textContent);

    if(nPh1!=7){
        document.getElementById('node1_ph').style.color = "yellow";
    }
    if(nPh2!=7){
        document.getElementById('node2_ph').style.color = "yellow";
    }
    
    if(nPh3!=7){
        document.getElementById('node3_ph').style.color = "yellow";
    }
    /**
    if(nPh4!=7){
        document.getElementById('node4_ph').style.color = "yellow";
    }
    **/

    //CHECK KLASIFIKASI KELEMBABAN TANAH
    var nLembab1 = parseFloat(document.getElementById('node1_lembab').textContent);
    var nLembab2 = parseFloat(document.getElementById('node2_lembab').textContent);
    var nLembab3 = parseFloat(document.getElementById('node2_lembab').textContent);
    //var nLembab4 = parseFloat(document.getElementById('node2_lembab').textContent);

    if(nLembab1<40 || nLembab1>62){
        document.getElementById('node1_lembab').style.color = "yellow";
    }
    if(nLembab2<40 || nLembab2>62){
        document.getElementById('node2_lembab').style.color = "yellow";
    }
    
    if(nLembab3<40 || nLembab3>62){
        document.getElementById('node3_lembab').style.color = "yellow";
    }
    /**
    if(nLembab4<40 || nLembab4>62){
        document.getElementById('node4_lembab').style.color = "yellow";
    }
    **/

    //CHECK KLASIFIKASI SUHU TANAH
    var nSuhuTh1 = parseFloat(document.getElementById('node1_suhuTh').textContent);
    var nSuhuTh2 = parseFloat(document.getElementById('node2_suhuTh').textContent);
    var nSuhuTh3 = parseFloat(document.getElementById('node3_suhuTh').textContent);
    //var nSuhuTh4 = parseFloat(document.getElementById('node4_suhuTh').textContent);
  

    if(nSuhuTh1<10 || nSuhuTh1>30){
        document.getElementById('node1_suhuTh').style.color = "yellow";
    }
    if(nSuhuTh2<10 || nSuhuTh2>30){
        document.getElementById('node2_suhuTh').style.color = "yellow";
    }
    
    if(nSuhuTh3<10 || nSuhuTh3>30){
        document.getElementById('node3_suhuTh').style.color = "yellow";
    }
    /**
    if(nSuhuTh4<10 || nSuhuTh4>30){
        document.getElementById('node4_suhuTh').style.color = "yellow";
    }
    **/

    //CHECK KLASIFIKASI SUHU UDARA
    var nSuhu1 = parseFloat(document.getElementById('node1_suhu').textContent);
    var nSuhu2 = parseFloat(document.getElementById('node2_suhu').textContent);
    var nSuhu3 = parseFloat(document.getElementById('node3_suhu').textContent);
    //var nSuhu4 = parseFloat(document.getElementById('node4_suhu').textContent);
  

    if(nSuhu1<18 || nSuhu1>26){
        document.getElementById('node1_suhu').style.color = "yellow";
    }
    if(nSuhu2<18 || nSuhu2>26){
        document.getElementById('node2_suhu').style.color = "yellow";
    }
    
    if(nSuhu3<18 || nSuhu3>26){
        document.getElementById('node3_suhu').style.color = "yellow";
    }
    /**
    if(nSuhu4<18 || nSuhu4>26){
        document.getElementById('node4_suhu').style.color = "yellow";
    }
    **/

</script>
</body>
</html>
\end{lstlisting}

\begin{lstlisting}[language=PHP, caption=skripsi\_perangkat.blade.php]
<!DOCTYPE html>
<html lang="en">
<head>
    <meta charset="UTF-8">
    <meta name="viewport" content="width=device-width, initial-scale=1.0, shrink-to-fit=no">
    <link rel="icon" href="{{ asset('assets/unpar.png') }}">
    <link rel="stylesheet" href="https://maxcdn.bootstrapcdn.com/bootstrap/4.0.0/css/bootstrap.min.css" integrity="sha384-Gn5384xqQ1aoWXA+058RXPxPg6fy4IWvTNh0E263XmFcJlSAwiGgFAW/dAiS6JXm" crossorigin="anonymous">

    <script src="https://code.jquery.com/jquery-3.2.1.slim.min.js" integrity="sha384-KJ3o2DKtIkvYIK3UENzmM7KCkRr/rE9/Qpg6aAZGJwFDMVNA/GpGFF93hXpG5KkN" crossorigin="anonymous"></script>
    <script src="https://cdnjs.cloudflare.com/ajax/libs/popper.js/1.12.9/umd/popper.min.js" integrity="sha384-ApNbgh9B+Y1QKtv3Rn7W3mgPxhU9K/ScQsAP7hUibX39j7fakFPskvXusvfa0b4Q" crossorigin="anonymous"></script>
    <script src="https://maxcdn.bootstrapcdn.com/bootstrap/4.0.0/js/bootstrap.min.js" integrity="sha384-JZR6Spejh4U02d8jOt6vLEHfe/JQGiRRSQQxSfFWpi1MquVdAyjUar5+76PVCmYl" crossorigin="anonymous"></script>
    <script src='https://kit.fontawesome.com/a076d05399.js'></script>
    
    <title>Print Sensing | Skripsi</title>
    <link href="{{ asset('css/skripsi_mainstyle.css') }}" rel="stylesheet" type="text/css" >
</head>
<style>
    body{
        background-image: url("assets/under_maintenance.jpg") ;
        background-repeat: repeat-y;
        background-size: 100%;
        padding-left: 99px;
        padding-right: 108px;}
        
    .dropbtn {
          background-color: lightgrey;
          color: black;
          padding: 4px;
          font-size: 15px;
          border: none;
          cursor: pointer;
        }

        .dropbtn:hover, .dropbtn:focus {
          background-color: #EEEDED;
        }

        .dropdown {
          position: relative;
          display: inline-block;
        }

        .dropdown-content {
          display: none;
          position: absolute;
          background-color: #EEEDED;
          min-width: 160px;
          overflow: auto;
          box-shadow: 0px 8px 16px 0px rgba(0,0,0,0.2);
          z-index: 1;
        }

        .dropdown-content a {
          color: black;
          padding: 12px 16px;
          text-decoration: none;
          display: block;
        }

        .dropdown a:hover {background-color: #ddd;}

        .show {display: block;}
</style>
<body>
    @include('skripsi_header')
    
    <div style="position:center;margin-top:20px">
        <form action="/print-sensing/cari" method="GET">
            
            <a style="color:white">Mulai :</a> 
            <input placeholder="Date" name="cariAwal" class="textbox-n" type="text" onfocus="(this.type='date')" id="date" value="{{ old('cariAwal') }}">
            
            <a style="color:white; padding-left:10px">Sampai :</a>
            <input placeholder="Date" name="cariAkhir" class="textbox-n" type="text" onfocus="(this.type='date')" id="date" value="{{ old('cariAkhir') }}">
            
            <input type="submit" value="cari">
        </form>
        
                    
        
              
        <button onclick="window.print()" style="float:right;style="border-radius:5px;"">Print</button>
        
        <div class="dropdown" style="float:right;margin-right:10px">
          <button onclick="myFunction()" class="dropbtn">Urutkan <i class="fas fa-angle-down"></i> </button>
          <div id="myDropdown" class="dropdown-content">
            <a href="/print-sensing/sortkode">Kode Petak</a>
            <a href="/print-sensing">Waktu Sensing</a>
          </div>
        </div>
    </div>
    
    
    <table style="margin-top: 24px;float: left; width:100%;" id="table1">
    
      <tr style="background-color: rgba(0, 0, 0, 0.5)">
        <!--<th>ID Sensing</th>
        <th>Jenis Tanah</th> -->
        <th>Kode Petak</th> 
        <th>Waktu Sensing</th> 
        <th>Keasaman (pH)</th> 
        <th>Kelembaban Tanah</th> 
        <th>Kelembaban Udara</th> 
        <th>Suhu Tanah</th> 
        <th>Suhu Udara</th> 
      </tr>
      @foreach ($tanah1 as $tanah1)
      <tr>
        <!--<td>{{$tanah1->id_tanah}}</td>-->
        <!--<td>{{$tanah1->jenis_tanah}}</td>-->
        <td>{{$tanah1->kode_node}}</td>
        <td>{{$tanah1->waktu_sensing}}</td>
        <td>{{$tanah1->ph_tanah}}</td>
        <td>{{$tanah1->kelembaban_tanah}}</td>
        <td>{{$tanah1->kelembaban_udara}}</td>
        <td>{{$tanah1->suhu_tanah}}</td>
        <td>{{$tanah1->suhu_udara}}</td>
    
        
      </tr>
      @endforeach
    </table>
    
<script>
    /* When the user clicks on the button, 
    toggle between hiding and showing the dropdown content */
    function myFunction() {
      document.getElementById("myDropdown").classList.toggle("show");
    }

    // Close the dropdown if the user clicks outside of it
    window.onclick = function(event) {
      if (!event.target.matches('.dropbtn')) {
        var dropdowns = document.getElementsByClassName("dropdown-content");
        var i;
        for (i = 0; i < dropdowns.length; i++) {
          var openDropdown = dropdowns[i];
          if (openDropdown.classList.contains('show')) {
            openDropdown.classList.remove('show');
          }
        }
      }
    }
</script>
    
</body>

</html>
\end{lstlisting}

\begin{lstlisting}[language=PHP, caption=skripsi\_carapakai.blade.php]
<!DOCTYPE html>
<html lang="en">
<head>
    <meta charset="UTF-8">
    <meta name="viewport" content="width=device-width, initial-scale=1.0, shrink-to-fit=no">
    <link rel="icon" href="{{ asset('assets/unpar.png') }}">
    <link rel="stylesheet" href="https://maxcdn.bootstrapcdn.com/bootstrap/4.0.0/css/bootstrap.min.css" integrity="sha384-Gn5384xqQ1aoWXA+058RXPxPg6fy4IWvTNh0E263XmFcJlSAwiGgFAW/dAiS6JXm" crossorigin="anonymous">

    <script src="https://code.jquery.com/jquery-3.2.1.slim.min.js" integrity="sha384-KJ3o2DKtIkvYIK3UENzmM7KCkRr/rE9/Qpg6aAZGJwFDMVNA/GpGFF93hXpG5KkN" crossorigin="anonymous"></script>
    <script src="https://cdnjs.cloudflare.com/ajax/libs/popper.js/1.12.9/umd/popper.min.js" integrity="sha384-ApNbgh9B+Y1QKtv3Rn7W3mgPxhU9K/ScQsAP7hUibX39j7fakFPskvXusvfa0b4Q" crossorigin="anonymous"></script>
    <script src="https://maxcdn.bootstrapcdn.com/bootstrap/4.0.0/js/bootstrap.min.js" integrity="sha384-JZR6Spejh4U02d8jOt6vLEHfe/JQGiRRSQQxSfFWpi1MquVdAyjUar5+76PVCmYl" crossorigin="anonymous"></script>

    <title>Cara Pakai | Skripsi</title>
    <link href="{{ asset('css/skripsi_mainstyle.css') }}" rel="stylesheet" type="text/css" >
</head>
<style>
    body{
        background-image: url("assets/carapakai_bg.jpg") ;
        background-size: 100%;
        padding-left: 99px;
        padding-right: 108px;
        background-repeat: no-repeat;
        /* box-shadow: 0 4px 8px 0 rgba(0, 0, 0, 0.2), 0 6px 20px 0 rgba(0, 0, 0, 0.19); */
    }
</style>
<body style="background-color: #ececec41;">
    @include('skripsi_header')

    <h1>Aplikasi Pemantauan<br> 
        Kualitas Tanah Sawah<br>
        Berbasis WSN
    </h1>

    <hr>

    <h2 style="font-size: 14px;">
        Skripsi II<br>
    </h2>

    <div class="carapakaihead">
        <img src="assets/carapakai1.png" style="width:78%;height:78%;">
    </div>
    <div class="carapakaicontent">
        <h1>Cara Penggunaan</h1>
        <hr>
        <div class="cpisi" style="margin-top: 80px;">
            <img src="assets/cp1.jpg">
                <h1>Aktifkan Perangkat</h1>
                <a>Aktifkan seluruh node sensor (perangkat keras Arduino dan Raspberry) 
                   juga pastikan setiap sensor sensing pada setiap node sensor aktif 
                   ,untuk melakukan pemantauan kualitas tanah sawah yang diuji.
                </a>
        </div>


        <div class="cpisi">
            <img src="assets/cp2.jpeg">
                <h1>Sebar Perangkat</h1>
                <a>Sebarkan node sensor ke berbagai titik sesuai dengan topologi yang digunakan.
                    Penyebaran node sensor dilakukan pada satu bidang petak tanah sawah.
                    Untuk node sensor Raspberry akan terhubung langsung dengan laptop.
                </a>
        </div>

        <div class="cpisi">
            <img src="assets/cp3.jpg">
                <h1>Aktifkan Internet</h1>
                <a>Pastikan laptop yang terhubung dengan perangkat Rapsberry terhubung dengan internet.
                    Hasil sensing yang didapatkan oleh node sensor akan dikirimkan ke Rapsberry lalu disimpan
                    di internet (local-host).
                </a>
        </div>

        <div class="cpisi">
            <img src="assets/cp4.jpg">
                <h1>Lakukan Sensing</h1>
                <a>Untuk melakukan sensing, masuk ke halaman sensing pada website. Hasil sensing yang 
                    diambil oleh node sensor secara otomatis akan ditampilkan pada halaman tersebut.
                    Pengguna juga dapat menghentikan sensing ataupun memulai kembali sensing pada halaman tersebut.
                </a>
        </div>      
    </div>

    <a class="trademark">Pengembangan Aplikasi Pemantauan Kualitas Tanah Sawah Berbasis WSN</a>

</body>
</html>
\end{lstlisting}


\begin{lstlisting}[language=PHP, caption=web.php]
<?php
use App\Event\TaskEvent;

use Illuminate\Support\Facades\Route;

/*
|--------------------------------------------------------------------------
| Web Routes
|--------------------------------------------------------------------------
|
| Here is where you can register web routes for your application. These
| routes are loaded by the RouteServiceProvider within a group which
| contains the "web" middleware group. Now create something great!
|
*/

Route::get('/', function () {
    return view('skripsi_home');
});

Route::get('/cara-pakai', function () {
    return view('skripsi_carapakai');
});

Route::get('/print-sensing', function () {
    return view('skripsi_perangkat');
});

Route::get('/GIS', function () {
    return view('skripsi_gis');
});


Route::get('/print-sensing','PrintViewController@index');
Route::get('/print-sensing/cari','PrintViewController@cari');
Route::get('/print-sensing/sortkode','PrintViewController@sortkode');

Route::get('/GIS','GISViewController@index');

Route::get('/sensing','SensingViewController@index');

Route::get('/check-status','StatusViewController@index');
Route::get('/check-status/update','StatusViewController@update');
Route::put('/check-status/update', 'StatusViewController@update');

Route::get('event', function () {
    event(new TaskEvent('Hey How Are You'));
});


\end{lstlisting}


\begin{lstlisting}[language=PHP, caption=controller.php]
<?php

namespace App\Http\Controllers;

use Illuminate\Foundation\Auth\Access\AuthorizesRequests;
use Illuminate\Foundation\Bus\DispatchesJobs;
use Illuminate\Foundation\Validation\ValidatesRequests;
use Illuminate\Routing\Controller as BaseController;

class Controller extends BaseController
{
    use AuthorizesRequests, DispatchesJobs, ValidatesRequests;
}
\end{lstlisting}


\begin{lstlisting}[language=PHP, caption=StatusViewController.php]
<?php
    namespace App\Http\Controllers;
    use Illuminate\Http\Request;
    use DB;
    use App\Http\Requests;
    use App\Http\Controllers\Controller;

    class StatusViewController extends Controller {
        public function index(){
            $check1 = DB::select(
                'SELECT nama_node,status_node,status_sensing 
                    FROM `nodesensor` 
                WHERE kode_node= 1'
            );
            $check2 = DB::select(
                'SELECT nama_node,status_node,status_sensing 
                    FROM `nodesensor` 
                WHERE kode_node= 2'
            );
            $check3 = DB::select(
                'SELECT nama_node,status_node,status_sensing 
                    FROM `nodesensor` 
                WHERE kode_node= 3'
            );
            $check4 = DB::select(
                'SELECT nama_node,status_node,status_sensing 
                    FROM `nodesensor` 
                WHERE kode_node= 4'
            );
            $check5 = DB::select(
                'SELECT nama_node,status_node,status_sensing 
                    FROM `nodesensor` 
                WHERE kode_node= 5'
            );
            return view('skripsi_checkstatus',['check1'=>$check1,'check2'=>$check2,'check3'=>$check3,'check4'=>$check4,
            'check5'=>$check5]); 
        }
    
        
        public function update(){
        DB::select(
          'UPDATE 
                nodesensor 
            SET 
                status_node = "Offline" ,status_sensing = "Not Sensing" 
            WHERE
                kode_node != 5');
        
        return redirect('/check-status');
        }
    }
\end{lstlisting}

\begin{lstlisting}[language=PHP, caption=PrintViewController.php]
<?php
    namespace App\Http\Controllers;
    use Illuminate\Http\Request;
    use DB;
    use App\Http\Requests;
    use App\Http\Controllers\Controller;

    class PrintViewController extends Controller {
        public function index(){
			$tanah1 = DB::select
			    ('SELECT
				*
			      FROM
				tanah
			       ORDER BY
				waktu_sensing DESC
				LIMIT
				    30');
            
            return view('skripsi_perangkat',['tanah1'=>$tanah1]); 
        }
        
        public function cari(Request $request)
		{
			// menangkap data pencarian
			$cariAwal = $request->cariAwal;
			$cariAkhir = $request->cariAkhir;
	 
			// mengambil data dari table sensing sesuai pencarian data
			$tanah1 = DB::table('sensing')
			->where('waktu_sensing','>',$cariAwal)
			->where('waktu_sensing','<',$cariAkhir)
			->paginate();
	 
			// mengirim data pegawai ke view index
			return view('skripsi_perangkat',['tanah1' => $tanah1]);
	 
		}
		
	public function sortkode(){
	    $tanah1 = DB::select
			    ('SELECT
				*
			      FROM
				sensing
			       ORDER BY
				kode_node ASC,waktu_sensing DESC
			    ');
				    
	    return view('skripsi_perangkat',['tanah1'=>$tanah1]);
	}
	    
    }
\end{lstlisting}

\begin{lstlisting}[language=PHP, caption=SensingViewController.php]
<?php
    namespace App\Http\Controllers;
    use Illuminate\Http\Request;
    use DB;
    use App\Http\Requests;
    use App\Http\Controllers\Controller;

    class SensingViewController extends Controller {
        public function index(){
            $nodes1 = DB::select(
                'SELECT 
                    nodesensor.kode_node  , waktu_sensing ,nodesensor.nama_node,ph_tanah,kelembaban_tanah,suhu_tanah,suhu_udara 
                FROM
                    nodesensor JOIN sensing ON sensing.kode_node = nodesensor.kode_node JOIN tanah ON tanah.id_Tanah = nodesensor.kode_node
                WHERE
                    waktu_sensing IN (SELECT 
                        max(waktu_sensing)
                    FROM
                        sensing
                    WHERE
                        kode_node = 1)'
            );
            $nodes2 = DB::select(
               'SELECT 
                    nodesensor.kode_node  , waktu_sensing ,nodesensor.nama_node,ph_tanah,kelembaban_tanah,suhu_tanah,suhu_udara 
                FROM
                    nodesensor JOIN sensing ON sensing.kode_node = nodesensor.kode_node JOIN tanah ON tanah.id_Tanah = nodesensor.kode_node
                WHERE
                    waktu_sensing IN (SELECT 
                        max(waktu_sensing)
                    FROM
                        sensing
                    WHERE
                        kode_node = 2)'
            );
            $nodes3 = DB::select(
                'SELECT 
                    nodesensor.kode_node  , waktu_sensing ,nodesensor.nama_node,ph_tanah,kelembaban_tanah,suhu_tanah,suhu_udara 
                FROM
                    nodesensor JOIN sensing ON sensing.kode_node = nodesensor.kode_node JOIN tanah ON tanah.id_Tanah = nodesensor.kode_node
                WHERE
                    waktu_sensing IN (SELECT 
                        max(waktu_sensing)
                    FROM
                        sensing
                    WHERE
                        kode_node = 3)'
            );
            $nodes4 = DB::select(
                'SELECT 
                    nodesensor.kode_node  , waktu_sensing ,nodesensor.nama_node,ph_tanah,kelembaban_tanah,suhu_tanah,suhu_udara 
                FROM
                    nodesensor JOIN sensing ON sensing.kode_node = nodesensor.kode_node JOIN tanah ON tanah.id_Tanah = nodesensor.kode_node
                WHERE
                    waktu_sensing IN (SELECT 
                        max(waktu_sensing)
                    FROM
                        sensing
                    WHERE
                        kode_node = 4)'
            );
            return view('skripsi_sensing',['nodes1'=>$nodes1,'nodes2'=>$nodes2,'nodes3'=>$nodes3,'nodes4'=>$nodes4]); 
        }
    }
\end{lstlisting}


% \lstinputlisting[language=Java, caption=MyCode.java]{./Lampiran/MyCode.java} 



