\chapter{Kesimpulan dan Saran}
\label{chap:kesimpulan dan Saran}
\section{Kesimpulan}
Berdasarkan hasil penelitian yang dilakukan, diperoleh kesimpulan - kesimpulan sebagai berikut:
\begin{enumerate}
	\item Aplikasi pemantauan kualitas tanah sawah berbasis {\it Wireless Sensor Network} telah berhasil dibangun baik dari sisi aplikasi admin maupun aplikasi pengguna.
	\item Berdasarkan pengujian, aplikasi yang dibangun berhasil melakukan \textit{sensing} terhadap tanah sawah yang diuji, juga berhasil dalam mengirimkan data tersebut ke \textit{base station} dengan jumlah sensor node sebanyak dua. Hasil pengujian juga menunjukan bahwa \textit{base station} telah berhasil menerima data hasil \textit{sensing} yang dikirimkan, serta menyimpan data tersebut dan menampilkannya ke \textit{browser}. 
	\item Hasil pengujian juga memperlihatkan bahwa aplikasi admin di \textit{base station} telah berhasil mengirim beberapa opsi instruksi ke node sensor dan mendapatkan respon.
	\item Hasil pengujian lapangan juga menunjukan bahwa aplikasi yang dibangun mampu memperkirakan kualitas tanah sawah yang dilakukan pengamatan
\end{enumerate}

\section{Saran}
Berdasarkan hasil penelitian yang dilakukan, ada beberapa saran untuk pengembangan aplikasi sebagai berikut:
\begin{enumerate}
	\item Aplikasi ini menyimpan data hasil \textit{sensing} di basis data yang bersifat \textit{local}. Akan lebih baik jika hasil \textit{sensing} disimpan di internet (\textit{cloud}). Dengan menyimpan data di \textit{cloud}, akses data hasil \textit{sensing} dapat ditampilkan diberbagai jenis perangkat yang terhubung dengan internet. 
	\item Perangkat keras dari aplikasi yang dibangun disebar menggunakan rangkaian kayu sederhan untuk meletakan node. Akan lebih baik jika rangkaian kayu dapat lebih tinggi dan kabel sensor \textit{sensing} diperpanjang sampai mencapai tanah, sehingga proses penyebaran antar node dapat saling berjauhan tanpa adanya kendala sinyal yang terhalang oleh tanaman padi dalam proses pengiriman hasil sensing.
	\item Pemantauan kualitas tanah sawah yang dilakukan di skripsi ini dilakukan pada waktu siang hari dan pagi hari. Proses pemantauan kedepannya dapat dilakukan di waktu yang lebih variatif dengan kondisi cuaca yang berbeda-beda, seperti pada waktu subuh, sore, atau malam hari.
\end{enumerate}